\chapter{Numerical Study of Motion Compensation}


In the previous chapter of this thesis a motion compensation algorithm is proposed as a general algorithm, and then is specifically focused for IGRT. The method however is scarcely compared with standard IGRT 4D CBCT image reconstruction methods, and some questions about the reliability of the motion models arise. Obtaining accurate motion description of the patients is still one of the biggest challenges in 4D imaging, regardless of the method used. How accurate do this models need to be? Additionally, if the motion is previously known to high accuracy, is 4D imaging necessary at all? 

This chapter looks more specifically at these challenges and compares the motion compensated reconstruction to the now most commonly used methods in clinical IGRT. The aim of the work here is to supplement Chapter \ref{ch:motion} with further observations about the behaviour of the algorithm under less ideal numerical data.

The objectives of this chapter are twofold: Firstly, the improved image quality obtainable by iterative algorithms is highlighted, showing how 4D CBCT binning methods can be improved by using better reconstruction algorithms, even with low data. Secondly, the flexibility and error behaviour of the algorithm is studied. The algorithm will reconstruct the image without any motion artefacts if the motion is perfectly know, however respiratory motion is variable between patients and within the patients themselves. The error behaviour with uncertainties in the DVF crucial for the future possibility of the method in clinical cases, as while the motion correction method removes almost in its entirety any motion artefacts with very accurately known DVFs, obtaining very accurate motion models of patients is not possible. Thus, the performance of the method with low resolution and approximately accurate DVFs is studied in this work. Additionally, some proposed methods for 4D CBCT rely in computing DVFs and then deforming a high resolution image with them. This work also studies why using reconstruction for motion correction produces better results than deforming a static image.

\section{Methods to evaluate quality}
This chapter reconstruct 4D images in all frames, and compare them to the ground truth. The iterative algorithms used in this section are SART and ASD-POCS, relatively arbitrarily chosen. To demonstrate the flexibility of the method, more than one algorithm is presented, and SART is chosen because its a well understood and common algorithm, while ASD-POCS is chosen because its a more advanced algorithm with more complex constrains, however it is quite well known also. One would expect that more advanced and newer algorithms to work even better than these two, but using those may obscure the results of the analysis that this chapter attempt to study.

To evaluate the quality of the reconstruction, the tumour is going to be the focus, as  in the previous chapter. The metrics RMSE and UQI and Segmentation mismatch will also be used, however an additional metric to compute the binary shape location of the tumour will also be used.  Using the same tumour area, the tumour will be extracted using morphological operators on images, via binarization with Otsu's method, image dilation and erosion and labelling using connected components. The biggest segmented blob will be then used to compute the geometric center, and the euclidean distance between this and the ground truth will be used as metric of quality. This is due to CBCT not reconstructing HU units of images with the best accuracy, thus the quality of the result on image attenuation coefficient values is less important than the quality of the shape of the tumour. The attenuation coefficients are actually important for RT planning, however CBCT is mainly used to know the tumour shape and location on the treatment.

\section{Iterative Algorithms for 4D CBCT}
\section{Substandard Deformation Vector Fields}
\subsection{Coarse Deformation Vector Fields}
\subsection{Cropped Vector Fields}