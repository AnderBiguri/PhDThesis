\chapter{Lung Cancer, Radiotherapy and Computed Tomography}

\section{Lung Cancer and treatment}


Lung cancer is the most common cancer in the world in men, both in amount of cases and mortality. In women, is third in cases and second in mortality, after breast cancer (\cite{WCR2014}). Yearly deaths due to lung cancer go over 1.5 million (see figure \ref{fig:world} for incidence), having around 10\% of five year survival rate in developed countries, and much lower in developing countries (\cite{CRUK2014}). One over fourteen people has a lifetime risk of developing lung cancer (\cite{Harrisons2012}), in average between men and women. The high incidence and mortality rates has lead to a big amount of research in research fields from all disciplines, in order to push further the detection and treatment of the disease, with an output of over 23.000 lung cancer related research articles in prestigious journals in the last 10 years (\cite{Nature2015}). In addition, the  actual lung cancer treatment has been transformed from non-existent in the 70s to used worldwide (\cite{Comis2003}).


\begin{figure}[ht]
\begin{center}
\includegraphics[width=0.98\columnwidth]{worldmapGLOBOCAN.png}
\caption[Lung cancer incidence in the world]{Lung cancer incidence per country, age adjusted data. Map and data from {GLOBOCAN} (\cite{GLOBOCAN2010}).}
%IARC has proprietary rights to the materials on the Website. Publications/data made available by IARC/WHO enjoy copyright protection in accordance with the provisions of Protocol 2 of the Universal Copyright Convention. All rights are reserved. Materials (fact sheets, maps, estimates or data) may be used "as is" for research, educational or other non-commercial purposes, but the corresponding reference must be cited in all cases.
\label{fig:world}
\end{center}
\end{figure}



The treatment of lung cancer varies between different types, but there are four main techniques: Chemotherapy, lobectomy or pneumoctomy, radiotherapy (RT) and palliative care. Generally, in early stages of small cell lung cancer the typical treatment would consist in chemotherapy with radiotherapy, and then brain radiotherapy, as there is chance that the tumour would spread to the head when treated. If the tumour has been detected in a very early stage and has not spread to the lymph nodes a lobectomy may be performed, removing part of the lung. Usually this is followed by radiotherapy and chemotherapy to make sure the tumour is killed.
In the case of non-small cell lung cancer, in the first stages a lobectomy or a pneumoctomy (removal of the whole lung) may be performed. Generally radiotherapy and chemotherapy (less likely) are also performed. In the last stages of the cancer, usually the treatment is palliative care i.e. treatments to reduce the symptoms and relief pain \citep{CRUK2014b}.

In practically all stages of different lung cancer treatments, radiotherapy is extensively used. About 120.000 patients use radiotherapy in the UK every year. Radiotherapy aims to kill malignant cells using ionizing radiation, generally using photons. High energy photons (X-rays) ionize the atoms that are part of the DNA chain. In photon therapy, this happens due to the ionization of the water in the cells, that forms free radicals, such as hydroxil radicals, destroying the DNA of the cells and killing them.