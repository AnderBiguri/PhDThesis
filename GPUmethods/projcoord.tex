\documentclass{article}

\usepackage{tikz}
\usetikzlibrary{calc,3d,arrows,shapes}
\usepackage{tikz-3dplot}
\usepackage{amsmath}
\usepackage{amsfonts}
\usetikzlibrary{external}
\tikzexternalize % activate!
\begin{document}

\begin{figure}[H]

\tdplotsetmaincoords{70}{125} % rotate 60 degrees around x axis, then 105 degrees about z
\begin{tikzpicture}[>=stealth',           % arrow tip
                    tdplot_main_coords,   % Set view as defined by \tdplotsetmaincoords
                    scale=0.4               % scale
                    ]  
               
               
               
               
               
               
               
    
               
                    
   %%%%%%%Some inside figure preable, for variables that are going to be repated
   %Trasnlation of CUBE
   \newcommand{\Tx}{0} 
   \newcommand{\Ty}{0} 
   \newcommand{\Tz}{0}
   \coordinate (T) at (\Tx,\Ty,\Tz);
   % Cube coords
   
   \newcommand{\CubeSz}{3}
   \coordinate (P1) at (-\CubeSz*0,-\CubeSz*1.4142,-\CubeSz);
   \coordinate (P2) at (-\CubeSz*0,-\CubeSz*1.4142,\CubeSz);
   \coordinate (P3) at (-\CubeSz*1.4142,\CubeSz*0,\CubeSz);
   \coordinate (P4) at (-\CubeSz*1.4142,\CubeSz*0,-\CubeSz);
   \coordinate (P5) at (\CubeSz*1.4142,-\CubeSz*0,-\CubeSz);
   \coordinate (P6) at (\CubeSz*1.4142,-\CubeSz*0,\CubeSz);
   \coordinate (P7) at (\CubeSz*0,\CubeSz*1.4142,\CubeSz);
   \coordinate (P8) at (\CubeSz*0,\CubeSz*1.4142,-\CubeSz);
   

   \path (1,0,0);
\pgfgetlastxy{\cylxx}{\cylxy}
\path (0,1,0);
\pgfgetlastxy{\cylyx}{\cylyy}
\path (0,0,1);
\pgfgetlastxy{\cylzx}{\cylzy}
\pgfmathsetmacro{\cylt}{(\cylzy * \cylyx - \cylzx * \cylyy)/ (\cylzy * \cylxx - \cylzx * \cylxy)}
\pgfmathsetmacro{\ang}{atan(\cylt)}
\pgfmathsetmacro{\ct}{1/sqrt(1 + (\cylt)^2)}
\pgfmathsetmacro{\st}{\cylt * \ct}

   \newcommand{\cylrad}{5}


\foreach \ztint in {0,.1,...,0} {
\pgfmathsetmacro{\tint}{(\ztint/2 + .5)*100}
\fill[red,opacity=0.05] (\ct,\st-5,-\ztint+5) -- ++(0,0,-10) arc[start  angle=\ang,delta angle=-180,radius=\cylrad] -- ++(0,0,8) arc[start  angle=\ang+180,delta angle=180,radius=\cylrad];
}

\fill[red,opacity=0.05] (\ct,\st-5,5) -- ++(0,0,-10) arc[start angle=\ang,delta angle=180,radius=\cylrad] -- ++(0,0,8) arc[start angle=\ang+180,delta angle=-180,radius=\cylrad];



\draw[thick,->] (\CubeSz*1.41,\CubeSz*0,-\CubeSz)-- (\CubeSz*1.41-\CubeSz*1.41*0.3,\CubeSz*0+\CubeSz*1.41*0.3,-\CubeSz) ;
\node at (\CubeSz*1.6,\CubeSz*0.6,-\CubeSz){ ${x_p}$};
	
	
\draw[thick,->] (\CubeSz*1.41,\CubeSz*0,-\CubeSz)-- (\CubeSz*1.41-\CubeSz*1.41*0.4,-\CubeSz*0-\CubeSz*1.41*0.4,-\CubeSz) ;
\node at (\CubeSz*1.5,-\CubeSz*0.4,-\CubeSz){ ${y_p}$};

\draw[thick,->] (\CubeSz*1.41,\CubeSz*0,-\CubeSz)-- (\CubeSz*1.41,-\CubeSz*0,-\CubeSz+1.8) ;
\node at (\CubeSz*1.5,\CubeSz*0.3,-\CubeSz+2){ ${z_p}$};






   \newcommand{\DetSzOne}{5}
   \newcommand{\DetSzTwo}{0.2}

   \coordinate (D1) at (-\DetSzTwo,-\DetSzOne,-\DetSzOne);
   \coordinate (D2) at (-\DetSzTwo,-\DetSzOne,\DetSzOne);
   \coordinate (D3) at (-\DetSzTwo,\DetSzOne,\DetSzOne);
   \coordinate (D4) at (-\DetSzTwo,\DetSzOne,-\DetSzOne);
   \coordinate (D5) at (\DetSzTwo,-\DetSzOne,-\DetSzOne);
   \coordinate (D6) at (\DetSzTwo,-\DetSzOne,\DetSzOne);
   \coordinate (D7) at (\DetSzTwo,\DetSzOne,\DetSzOne);
   \coordinate (D8) at (\DetSzTwo,\DetSzOne,-\DetSzOne);
    \coordinate (T2) at (-12,-0.,-0.);
   % offset vector

   %Draw CUBE
    \draw[black,fill=blue,opacity=0.05] ($(P1)+(T)$) -- ($(P2)+(T)$) -- ($(P3)+(T)$) -- ($(P4)+(T)$) -- cycle;% -X face
	\draw[black,fill=blue!20,opacity=0.05] ($(P2)+(T)$) -- ($(P3)+(T)$) -- ($(P7)+(T)$) -- ($(P6)+(T)$) -- cycle;% +Z face
	\draw[black,fill=blue,opacity=0.05] ($(P1)+(T)$) -- ($(P2)+(T)$) -- ($(P6)+(T)$) -- ($(P5)+(T)$) -- cycle;% -Y face
	\draw[black,fill=blue,opacity=0.05] ($(P3)+(T)$) -- ($(P4)+(T)$) -- ($(P8)+(T)$) -- ($(P7)+(T)$) -- cycle;% Right Face
	\draw[black,fill=blue,opacity=0.1] ($(P1)+(T)$) -- ($(P4)+(T)$) -- ($(P8)+(T)$) -- ($(P5)+(T)$) -- cycle;% Front Face
	\draw[black,fill=blue,opacity=0.05] ($(P5)+(T)$) -- ($(P6)+(T)$) -- ($(P7)+(T)$) -- ($(P8)+(T)$) -- cycle;% Top Face
	
	%image
	\node at ($(P2)+(T)+(0,0,0.5)$){\Large $\mathbb{I}$};
	\node at ($(D2)+(T2)+(0,0,0.5)$){\Large $\mathbb{D}$};
	

	
	
	

  

    \draw[black,fill=black,opacity=0.05] ($(D1)+(T2)$) -- ($(D2)+(T2)$) -- ($(D3)+(T2)$) -- ($(D4)+(T2)$) -- cycle;% -X face
	\draw[black,fill=black!20,opacity=0.05] ($(D2)+(T2)$) -- ($(D3)+(T2)$) -- ($(D7)+(T2)$) -- ($(D6)+(T2)$) -- cycle;% +Z face
	\draw[black,fill=black,opacity=0.1] ($(D1)+(T2)$) -- ($(D2)+(T2)$) -- ($(D6)+(T2)$) -- ($(D5)+(T2)$) -- cycle;% -Y face
	\draw[black,fill=black!20,opacity=0.05] ($(D3)+(T2)$) -- ($(D4)+(T2)$) -- ($(D8)+(T2)$) -- ($(D7)+(T2)$) -- cycle;% Right Face
	\draw[black,fill=black,opacity=0.1] ($(D1)+(T2)$) -- ($(D4)+(T2)$) -- ($(D8)+(T2)$) -- ($(D5)+(T2)$) -- cycle;% Front Face
	\draw[black,fill=black,opacity=0.05] ($(D5)+(T2)$) -- ($(D6)+(T2)$) -- ($(D7)+(T2)$) -- ($(D8)+(T2)$) -- cycle;% Top Face


\draw[black,fill=red,opacity=0.5] ($(0,1,3)+(T2)$) -- ($(0,1,1)+(T2)$) -- ($(0,3,1)+(T2)$) -- ($(0,3,3)+(T2)$) -- cycle;% Top Face

%
%\foreach \x in {1,1.25,1.5,1.75,2,2.25,2.5,2.75,3}
%	\draw[black,thin] (-12,1,\x)--(-12,3,\x);
%	\foreach \x in {1,1.25,1.5,1.75,2,2.25,2.5,2.75,3}
%	\draw[black,thin] (-12,\x,1)--(-12,\x,3);
%	% draw kernel
%	
    \draw[gray,dashed] (10,0,0)--($(0,1,3)+(T2)$);
    \draw[gray,dashed] (10,0,0)--($(0,1,1)+(T2)$);
    \draw[gray,dashed] (10,0,0)--($(0,3,1)+(T2)$);
    \draw[gray,dashed] (10,0,0)--($(0,3,3)+(T2)$);
    
    \draw[gray] (5,0.2273,0.2273*3)--(-5,0.6818,0.6818*3);
    \draw[gray] (5,0.2273,0.2273)--(-5,0.6818,0.6818);
    \draw[gray] (5,0.2273*3,0.2273*3)--(-5,0.6818*3,0.6818*3);
    \draw[gray] (5,0.2273*3,0.2273)--(-5,0.6818*3,0.6818);
    

%        \draw[fill=black]   (5,0.2273,0.2273*3) circle (0.2);
%        \draw[fill=black]   (5,0.2273,0.2273)circle (0.2);
%        \draw[fill=black]   (5,0.2273*3,0.2273*3)circle (0.2);
%        \draw[fill=black]   (5,0.2273*3,0.2273)circle (0.2);
        
\node[draw=none,shape=circle,fill, inner sep=1pt,gray] (d1) at (5,0.2273,0.2273){};  % circle
\node[draw=none,shape=circle,fill, inner sep=1pt,gray] (d1) at (5,0.2273*3,0.2273){};  % circle
\node[draw=none,shape=circle,fill, inner sep=1pt,gray] (d1) at (5,0.2273,0.2273*3){};  % circle
\node[draw=none,shape=circle,fill, inner sep=1pt,gray] (d1) at (5,0.2273*3,0.2273*3){};  % circle

\node[draw=none,shape=circle,fill, inner sep=1pt,gray] (d1) at (-5,0.6818,0.6818){};  % circle
\node[draw=none,shape=circle,fill, inner sep=1pt,gray] (d1) at (-5,0.6818*3,0.6818){};  % circle
\node[draw=none,shape=circle,fill, inner sep=1pt,gray] (d1) at (-5,0.6818,0.6818*3){};  % circle
\node[draw=none,shape=circle,fill, inner sep=1pt,gray] (d1) at (-5,0.6818*3,0.6818*3){};  % circle

%%
%% block path
%\draw[black,fill=gray,opacity=0.3] (3,0.3182,0.3182)--(3,0.3182,0.3182*3)--(3,0.3182*3,0.3182*3)--(3,0.3182*3,0.3182) -- cycle;
%\draw[black,fill=gray,opacity=0.3] (-3,0.5909,0.5909)--(-3,0.5909,0.5909*3)--(-3,0.5909*3,0.5909*3)--(-3,0.5909*3,0.5909) -- cycle;
% \draw[black,fill=gray,opacity=0.3] (3,0.3182,0.3182)--(3,0.3182,0.3182*3)--(-3,0.5909,0.5909*3)--(-3,0.5909,0.5909) -- cycle;
% \draw[black,fill=gray,opacity=0.3] (3,0.3182*3,0.3182)--(3,0.3182*3,0.3182*3)--(-3,0.5909*3,0.5909*3)--(-3,0.5909*3,0.5909) -- cycle; 
% \draw[black,fill=gray,opacity=0.3] (3,0.3182,0.3182)--(3,0.3182*3,0.3182)--(-3,0.5909*3,0.5909)--(-3,0.5909,0.5909) -- cycle; 
% \draw[black,fill=gray,opacity=0.3] (3,0.3182*3,0.3182*3)--(3,0.3182,0.3182*3)--(-3,0.5909,0.5909*3)--(-3,0.5909*3,0.5909*3) -- cycle; 
% 
% \draw[black,fill=red,opacity=0.3] (0,0.4546,0.4546)--(0,0.4546,0.4546*3)--(0,0.4546*3,0.4546*3)--(0,0.4546*3,0.4546) -- cycle;
% 
% 
% 
%   \draw[thick,->] (0,-4,4)node[anchor=south east, align=left]{Concurrent \\ memory read} -- (0,0.4546,0.4546*3) ;
%   
% \draw[thick,->] (0,4,-4)node[anchor=north west, align=left]{Memory used \\by a single\\ thread block} -- (-1.51,0.5*3,0.5) ;
 
 
% \draw[<->](-12,3.2,1)--(-12,3.2,3);
% \node at (-12,4.5,2.2){${\scriptsize divV}$};
% 
%  \draw[<->](-12,1,3.2)--(-12,3,3.2);
% \node at (-12,2.2,3.9){${\scriptsize divU}$};
% 
 
 
 
	 %% Source
	 
    \newcommand{\rotatedtangentangle}[1]{%
    % find directions of projection
    \path[tdplot_rotated_coords] (1,0,0);
    \pgfgetlastxy{\axisxx}{\axisxy}
    \path[tdplot_rotated_coords] (0,1,0);
    \pgfgetlastxy{\axisyx}{\axisyy}
    \path[tdplot_rotated_coords] (0,0,1);
    \pgfgetlastxy{\axiszx}{\axiszy}
    % angle of tangent
    \pgfmathsetmacro{\rtang}{atan(-\axiszy/\axiszx)+180}
    \pgfmathsetmacro{\angkorr}{atan(\axisyy/\axisyx)/2}

    \pgfmathsetmacro{#1}{\rtang+\angkorr}
}%      
    \tdplotsetthetaplanecoords{90} % create rotated frame
    \rotatedtangentangle{\tangent} % compute tanget angle
    % shift rotated frame to center of cylinder
    \coordinate (shift) at (10,0,0);
    \tdplotsetrotatedcoordsorigin{(shift)}
    % draw cylinder
    \begin{scope}[tdplot_rotated_coords]
        \draw[fill=red!30]
            (0,0,-1.5) ++(\tangent:0.5) -- ++(0,0,1.5) arc (\tangent:\tangent-180:0.5) -- ++(0,0,-1.5);
        \draw[fill=red!30] (0,0,-1.5) circle [radius=0.5];
        
    \end{scope}
    \node at (10.4,0,0) {S};
	
    
\end{tikzpicture}
\centering

\caption{Geometry of CBCT}
\end{figure}

\end{document}
